\subsection{Próximos pasos}

Este trabajo acerca la idea de implementar un sistema de seguridad en las formaciones ferroviaria, pero para poder cumplir ese objetivo queda pendiente la realización de otros proyectos que se van a beneficiar y desprender de este. Algunos de los próximos pasos son los siguientes:

\begin{itemize}
    \item Finalización del armado del software para la central operativa. Esto debe contar con interfaces de monitoreo, bases de datos donde almacenar la información y la capacidad de interactuar con distintas formaciones. 

    \item Pruebas sobre laboratorios de Trenes Argentinos utilizando los sistemas externos reales para las entradas y salidas del sistema.

    \item Integración del microcontrolador y los circuitos utilizados de la placa Nucleo 144 sobre la placa de procesamiento diseñada para el SAL/T.

    \item Reemplazo de algunos componentes por su versión de mayor seguridad y robustez como pueden ser los relés de seguridad, los módems 4G - Wi-Fi o la antena para la conectividad GPS. Estos cambios fueron previstos para una próxima versión por lo que el diseño de hardware permite estas modificaciones de componentes sin cambios en el diseño del PCB. 

    \item Fabricación de un gabinete con la robustez necesaria para su instalación en una formación ferroviaria considerando las vibraciones e interferencia electromagnética presentes en el entorno de operación.

    \item Diseño e implementación del sistema de adaptación de la señal del generador de impulsos ópticos a una comunicación RS485.

    \item Diseño e implementación de una fuente de alimentación externa para el SAL/T que permite la reducción de tensión de las baterías de la formación en el rango de 72-110 VDC a 5VDC. 

    
\end{itemize}