\subsection{Conclusiones generales}

En este trabajo se logró diseñar y armar un nuevo prototipo de un sistema de aislado limitado / total que cumple con todos los requerimientos acordados para el proyecto. Esto permite acercar el objetivo de realizar la fabricación de un SAL/T de manera nacional y mejorar la seguridad en la operación de las formaciones ferroviarias causando una mejora directa en la vida de las personas que utilizan este medio de transporte de manera cotidiana. \\

En este prototipo, se implementó un hardware más moderno que su versión anterior, pero sin dejar de reutilizar muchas de sus técnicas de diseño y lógicas de circuito. Esto permitió que este prototipo agregue funcionalidades y mejoras sobre el sistema sin dejar de contar con la robustez del prototipo predecesor. En cuanto al firmware, se logró implementar una lógica del sistema utilizando un sistema operativo en tiempo real que permita orquestar de manera eficiente los recursos y los tiempos de respuesta de cada uno de los subsistemas incluidos en el SAL/T. \\ 

Las pruebas integrales realizadas simulando todos los dispositivos externos propios de una formación ferroviaria permiten confirmar que el prototipo cumple con el comportamiento esperado en todas las situaciones especificadas en los requerimientos. Sin embargo, para poder validar el correcto funcionamiento del sistema en condiciones operativas de una formación ferroviaria, queda pendiente realizar pruebas conectadas a los sistemas reales de uso del SAL/T. Además de las potenciales diferencias con el banco de pruebas utilizado, es esperable que algunas funcionalidades o decisiones de implementación hayan quedado alejadas o diferentes a las imaginadas por el autor del pliego de especificaciones de Trenes Argentinos, ya que durante todo el proyecto no se logró mantener una iteración constante con la empresa para validar las decisiones de diseño y, si bien responde a los requerimientos del pliego, la interpretación de este puede llevar a distintos enfoques o pretensiones. \\ 

La etapa de relevamiento de los requerimientos y definición del alcance del proyecto resultó la etapa más relevante del trabajo porque todas las tareas subsiguientes se desprenden de esta primera etapa. Luego, el diseño de hardware significó la mayor carga de horas de trabajo por su naturaleza de incorporación de todos los requerimientos del proyecto y rigidez que conlleva la fabricación de un hardware incorrecto. Por esto, se hizo una primera etapa de diseño integral de la solución donde se estableció de manera modular los distintos subsistemas del SAL/T y la comunicación de cada uno con el controlador principal. Luego, se diseñaron e implementaron de manera independiente cada uno de los módulos respetando las interfaces definidas previamente. De esta manera se logró realizar un proceso ordenado de diseño desde lo general a lo particular. \\ 

Este trabajo tiene como mayor aporte la fabricación de un prototipo que considera todos los requerimientos para armar un sistema de seguridad productivo para instalar en las formaciones ferroviarias de la empresa Trenes Argentinos. El detalle de su diseño y consideraciones queda plasmado en este documento lo que permite entender y reutilizar una gran parte del proyecto al momento de realizar una versión productiva. \\

A nivel personal, este trabajo permitió consolidar y poner en práctica de manera integral gran parte de los conocimientos adquiridos durante la carrera de grado de ingeniería electrónica realizando un proyecto que atraviesa las etapas de relevamiento de los requerimientos, diseño de hardware, diseño de firmware, fabricación, implementación, verificación, diseño y ejecución de pruebas y presentación del proyecto mismo.