
\vspace*{\fill}

\begin{center}

{\huge \textit{Resumen}}


\vspace{1.5cm}

\begin{adjustwidth}{2cm}{2cm}
\begin{center}
    

        
Este trabajo presenta el diseño e implementación de un prototipo para un sistema de aislado limitado total para formaciones ferroviarias (SAL/T) basado en un pliego de especificaciones de la empresa Trenes Argentinos Operaciones y en una primera versión del prototipo realizado anteriormente en el contexto del Grupo de Investigación en Calidad y Seguridad de las Aplicaciones Ferroviarias (GICSAFe). En esta versión, se agregan nuevas funcionalidades, se actualiza el hardware utilizado y se mejora la portabilidad de su firmware. \\

\vspace{.8cm}

Este sistema permite aumentar la seguridad de los trenes ante una falla en alguno de sus sistemas instrumentados de seguridad interfiriendo las señales de corte de tracción y freno de emergencia de la formación y aislando el subsistema en fallo. El SAL/T considera múltiples fuentes de medición de la velocidad de la formación y la comunicación con una central operativa para determinar cómo debe intervenir estas señales críticas para llevar a la formación a un estado seguro de manera controlada.  \\ 

\vspace{.8cm}

A lo largo del proyecto, se aplicaron conocimientos de electrónica general, diseño de circuitos, diseño de placas impresas, utilización de protocolos de comunicación e implementación de firmware basado en sistemas operativos en tiempo real.


\end{center}
\end{adjustwidth}

\end{center}

\vspace{8cm}

\vspace*{\fill}

