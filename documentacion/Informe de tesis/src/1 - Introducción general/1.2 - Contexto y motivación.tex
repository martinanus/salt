
\subsection{Contexto y motivación}

El Grupo de Investigación en Calidad y Seguridad de las Aplicaciones Ferroviarias (GICSAFe) fue creado en 2017 en el marco del Consejo Nacional de Investigaciones Científicas y Técnicas (CONICET) de la República Argentina. En el grupo participan investigadores, docentes y alumnos de diferentes instituciones públicas argentinas quienes realizan desarrollos de sistemas electrónicos e informáticos para aplicaciones ferroviarias relacionadas con la seguridad. Muchos de los prototipos desarrollados son instalados y entregados junto a toda la documentación, respetando normas internacionales de seguridad, para luego transferir el derecho de uso, fabricación y mantenimiento a los clientes. \\



La empresa estatal Trenes Argentinos Operaciones mencionó en reiteradas oportunidades la disconformidad con la situación actual de las formaciones ferroviarias frente a algún fallo en los Sistemas Instrumentados de Seguridad (SIS) porque, al entrar alguno de estos en fallo, obliga a la formación a ser remolcada o, lo que sucede más frecuentemente en la práctica, a continuar hasta la próxima estación para descender a los pasajeros deshabilitando todos los SIS de la formación. Esto presenta una situación de riesgo porque la formación pasa a depender fuertemente de la manualidad del conductor sin ningún sistema de seguridad monitoreando y asistiendo la conducción. \\

Al mismo tiempo, la empresa no encuentra alternativas comerciales para satisfacer las necesidades de un Sistema de Aislado Limitado / Total ya que necesita integrar distintos SIS y fuentes de medición que no son necesariamente del mismo fabricante ni con especificaciones genéricas que permitan su fácil integración.  \\

Por estos motivos, Trenes Argentinos solicitó en el año 2017 al grupo CONICET-GICSAFe que diseñara un SAL/T a medida para cumplir con los requerimientos específicos de sus formaciones. El ingeniero Iván Di Vito, en ese momento estudiante de la carrera de Ingeniería Electrónica en la UBA y allegado al grupo de investigación tomó el proyecto que finalizaría en el año 2019 con la entrega de un prototipo funcional del SAL/T luego de varias pruebas de campo en los talleres de Trenes Argentinos junto a toda la documentación requeridas por las normas de seguridad ferroviarias para este tipo de desarrollos. El trabajo realizado incluyó el relevamiento de los requerimientos del proyecto, el desarrollo conceptual de la solución propuesta, el diseño e implementación del hardware y firmware, la puesta en marcha y realización de pruebas de campo de un prototipo funcional. El mismo se realizó de acuerdo a la norma UNE-EN 50126, aplicando técnicas de patrones de diseño de hardware y software \cite{patrones}. Este trabajo fue publicado por el Congreso Argentino de Sistemas Embebidos (CASE) en el año 2019 en la categoría de artículo \cite{salt_case}. \\

En el año 2021, dos estudiantes de la carrera de Ingeniería Electrónica de la UBA, Fernando Iglesias y Matias Sambrizzi, emprendieron un proyecto para el desarrollo de la central operativa del SAL/T en el mismo contexto  de acuerdo entre Trenes y CONICET-GICSAFe. El objetivo del proyecto fue desarrollar el software de una central operativa que permita administrar y configurar de forma remota dispositivos de supervisión de seguridad de formaciones ferroviarias SAL/T. Para esto, se implementó un servicio de monitoreo y control que permite visualizar en tiempo real los datos recibidos, enviar comandos de control a los dispositivos activos, gestionar la configuración de dispositivos, la base de datos, implementar los componentes de seguridad necesarios para las comunicaciones e información almacenada y gestionar los usuarios y perfiles. \\


Sin embargo, antes de requerir una versión productiva y reproducible en cantidades, se determinó que faltaban implementar algunas funcionalidades adicionales para que se pueda instalar un SAL/T en las formaciones. El pliego de especificaciones técnicas para la versión completa del SAL/T se encuentra en disponible en \cite{spec}. \\

Los cambios más relevantes mencionados en el documento que dan lugar a rediseñar el prototipo antes de producirlo en masa son: 
\begin{itemize}
    \item Utilización de un microcontrolador de mayor capacidad y más vigente
    \item Generación de un firmware más portable basado en un sistema operativo de tiempo real
    \item Aislamiento del SIS en falla en Modo Aislado Limitado sin interrumpir el funcionamiento del resto de los SIS
    \item Selección de distintos perfiles intermitentes desde el panel frontal del SAL/T
    \item Conector de zona para poder determinas distintas zonas de tránsito para la formación y activar cambios en otros sistemas de la formación
    \item Registro interno de eventos para cambios de estado de SIS, activación de modos, excesos de velocidad y activación de señales de Corte de Tracción y Freno de Emergencia
\end{itemize}

