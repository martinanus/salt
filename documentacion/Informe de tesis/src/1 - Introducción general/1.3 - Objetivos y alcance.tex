\subsection{Objetivos y alcance}

En este trabajo se busca hacer un rediseño del prototipo del SAL/T implementando algunas mejoras sugeridas por Trenes Argentinos en el pliego de especificaciones técnicas ET.SO.No 046/18–E3 \cite{spec} del 2022, utilizando un microcontrolador y componentes más modernos y un firmware basado en un sistema operativo de tiempo real que resulte más mantenible que el firmware de la primera versión del SAL/T desarrollada por Ivan Di Vito. Se busca diseñar una solución para aumentar la seguridad en las formaciones ferroviarias ante fallas en los Sistemas Instrumentados de Seguridad mediante un mejor control y monitoreo de velocidad de la formación en tiempo real. La solución debe ser compatible con los mecanismos de seguridad y sistemas instalados actualmente en las formaciones de Trenes Argentinos. Los principales objetivos que se persiguen son:


\begin{itemize}
    \item Rediseñar el sistema SAL/T a partir de los trabajos existentes y el pliego de especificaciones técnicas propuestos por Trenes Argentinos.
    \item Rediseñar el hardware del SAL/T utilizando componentes vigentes en el mercado.
    \item Generar un firmware portable basado en un sistema operativo de tiempo real.
    \item Construir un prototipo del proyecto y realizar pruebas de funcionamiento en un entorno similar al real.
    \item Favorecer el desarrollo de tecnología nacional.
\end{itemize}


El proyecto consiste en diseñar, desarrollar e implementar un nuevo prototipo del SAL/T teniendo en cuenta que se deben agregar nuevas funcionalidades requeridas, que el hardware debe ser actualizado con componentes vigentes y disponibles actualmente en el mercado internacional y que el firmware desarrollado debe estar basado en un sistema operativo de tiempo real que resulte portable. \\

Para establecer los requerimientos del proyecto, se contempló el prototipo preexistente del SAL/T y el pliego de especificaciones técnicas que generó Trenes Argentinos \cite{spec} teniendo en consideración las limitaciones de tiempos y presupuesto que hace sentido para un proyecto de fin de carrera de grado. \\

Los cambios más relevantes mencionados en el pliego de especificaciones técnicas \cite{spec} que dan lugar a rediseñar el prototipo antes de producirlo en masa son los siguientes:

\begin{itemize}
    \item Utilización de un microcontrolador de mayor capacidad y más vigente.
    \item Generación de un firmware más mantenible basado en un sistema operativo de tiempo real.
    \item Aislamiento del SIS en falla en Modo Aislado Limitado sin interrumpir el funcionamiento del resto de los SIS.
    \item Selección de distintos perfiles intermitentes desde el panel frontal del SAL/T.
    \item Conector de zona para poder determinas distintas zonas de tránsito para la formación y activar cambios en otros sistemas de la formación.
    \item Registro interno de eventos para cambios de estado de SIS, activación de modos, excesos de velocidad y activación de señales de Corte de Tracción y Freno de Emergencia.
\end{itemize}


Una vez finalizado el diseño del sistema, el prototipo debe ser fabricado sobre una placa PCB y se deben montar todos los componentes para su correcto funcionamiento. Se deberán realizar pruebas de laboratorio en las condiciones de funcionamiento para verificar que se cumplan todos los requerimientos establecidos. 


