\subsection{Objetivos y alcance}

Se busca diseñar una solución para aumentar la seguridad en las formaciones ferroviarias ante fallas en los Sistemas Instrumentados de Seguridad mediante un mejor control y monitoreo de velocidad de la formación en tiempo real. La solución debe ser compatible con los mecanismos de seguridad y sistemas instalados actualmente en las formaciones de Trenes Argentinos. Los principales objetivos que se persiguen son:


\begin{itemize}
    \item Rediseñar el sistema SAL/T a partir de los trabajos existentes y el pliego de especificaciones técnicas propuestos por Trenes Argentinos 
    \item Rediseñar el hardware del SAL/T utilizando componentes vigentes en el mercado
    \item Generar un firmware portable basado en un sistema operativo de tiempo real
    \item Construir un prototipo del proyecto y realizar pruebas de funcionamiento en un entorno similar al real
    \item Favorecer el desarrollo de tecnología nacional 
\end{itemize}


El proyecto consiste en diseñar, desarrollar e implementar un nuevo prototipo del SAL/T teniendo en cuenta que se deben agregar nuevas funcionalidades requeridas, que el hardware debe ser actualizado con componentes vigentes y disponibles actualmente en el mercado internacional y que el firmware desarrollado debe estar basado en un sistema operativo de tiempo real que resulte portable. \\

Para establecer los requerimientos del proyecto, se contempló el prototipo preexistente del SAL/T y el pliego de especificaciones técnicas que generó Trenes Argentinos teniendo en consideración las limitaciones de tiempos y presupuesto que hace sentido para un proyecto de fin de carrera de grado. \\

Una vez finalizado el diseño del sistema, el prototipo debe ser fabricado sobre una placa PCB y se deben montar todos los componentes para su correcto funcionamiento. Se deberán realizar pruebas de laboratorio en las condiciones de funcionamiento para verificar que se cumplan todos los requerimientos establecidos. 