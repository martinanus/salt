



\subsection{Contexto y motivación}

El Grupo de Investigación en Calidad y Seguridad de las Aplicaciones Ferroviarias (GICSAFe) fue creado en 2017 en el marco del Consejo Nacional de Investigaciones Científicas y Técnicas (CONICET) de la República Argentina \cite{gicsafe}. En el grupo participan investigadores, docentes y alumnos de diferentes instituciones públicas argentinas quienes realizan desarrollos de sistemas electrónicos e informáticos para aplicaciones ferroviarias relacionadas con la seguridad. \\

Las formaciones ferroviarias cuentan con varios Sistemas Instrumentados de Seguridad (SIS) a bordo con el objetivo de supervisar el correcto funcionamiento de los subsistemas críticos, como por ejemplo, la protección de coche a la deriva, el sistema de hombre vivo o la seguridad de puertas \cite{salt_paper}. \\

Estos sistemas están diseñados para que, ante la detección de una falla crítica, se produzca la detención de la formación de manera inmediata. Esto se logra controlando las señales de Corte de Tracción (CT) y de Freno de Emergencia (FE) \cite{salt_paper}. En esta situación, la formación puede desactivar todos los SIS bajo estricta supervisión y continuar desplazándose con el control manual del conductor hasta una estación cercana para descender a los pasajeros y luego a un taller para ser reparada. \\

El Sistema de Aislado Limitado/Total, por sus siglas SAL/T, es un equipo que permite una circulación controlada y más segura de la formación cuando uno de los SIS se encuentra en falla. Al activar el Modo Aislado Limitado (MAL), el SAL/T va a deshabilitar solamente el SIS que está en fallo y controlar las señales de CT y FE para poder continuar con la circulación de la formación. Para hacerlo de manera segura, el SAL/T va a contar con múltiples fuentes de medición de la velocidad de la formación y va a activar las señales de CT y FE para frenarla en caso de que se excedan los límites estipulados. En el Modo Aislado Total (MAT), se libera la velocidad de precaución y puede ser aplicado solamente por personal superior en el caso de que la formación se encuentre sin pasajeros y muy alejada del taller ferroviario. La activación de los modos MAL y MAT debe ser registrada por el SAL/T, así como también cualquier falla detectada, cambio de configuración en las velocidades permitidas o cualquier evento significativo para la seguridad de la formación \cite{salt_paper}.\\

Además, el SAL/T cuenta con otras funcionalidades como la comunicación con el conductor a través de una interfaz donde se muestra el estado de los SIS, la velocidad de circulación de la formación y un indicador de la zona donde se encuentra la formación. En caso de no contar con una medición de velocidad válida y activar el MAL, el SAL/T entra en Modo Intermitente (MI) y controla la formación activando y desactivando la tracción y el freno acorde a los perfiles de tiempos preconfigurados según las características de la formación para no exceder ciertas velocidades de precaución. Desde el panel formal, el conductor puede seleccionar distintos perfiles intermitentes.  También, el SAL/T tiene la posibilidad de conectarse con una central operativa para informar su estado y recibir instrucciones \cite{salt_paper}. \\


La empresa estatal Trenes Argentinos Operaciones \cite{trenes_arg} mencionó en reiteradas oportunidades la disconformidad con la situación actual de las formaciones ferroviarias frente a algún fallo en los Sistemas Instrumentados de Seguridad porque, al entrar alguno de estos en fallo,
la formación se detiene y su tracción se bloquea aún si está lejos de una estación, en un lugar inseguro para que los pasajeros desciendan. En la práctica, se deshabilitan todos los SIS de la formación y se conduce hasta la próxima estación para descender a los pasajeros. Esto presenta una situación de riesgo porque la formación pasa a depender fuertemente del criterio del conductor sin ningún sistema de seguridad monitoreando y asistiendo la conducción \cite{salt_paper}. \\

Al mismo tiempo, la empresa no encuentra alternativas comerciales para satisfacer las necesidades de un Sistema de Aislado Limitado / Total, que permita deshabilitar el bloqueo generado por algún SIS en particular, ya que necesita integrar distintos SIS y fuentes de medición que no son necesariamente del mismo fabricante ni con especificaciones genéricas que permitan su fácil integración.  \\

Por estos motivos, Trenes Argentinos solicitó en el año 2017 al grupo CONICET-GICSAFe que diseñara un SAL/T a medida para cumplir con los requerimientos específicos de sus formaciones. El ingeniero Iván Di Vito, en ese momento estudiante de la carrera de Ingeniería Electrónica en la UBA y allegado al grupo de investigación, tomó el proyecto que finalizó en el año 2019 con la entrega de un prototipo funcional del SAL/T luego de varias pruebas de campo en los talleres de Trenes Argentinos junto a toda la documentación requerida por las normas de seguridad ferroviarias para este tipo de desarrollos \cite{salt_ivan}. El trabajo realizado incluyó el relevamiento de los requerimientos del proyecto, el desarrollo conceptual de la solución propuesta, el diseño e implementación del hardware y firmware, la puesta en marcha y realización de pruebas de campo de un prototipo funcional. El mismo se realizó de acuerdo a la norma UNE-EN 50126 \cite{norma_50126}, aplicando técnicas de patrones de diseño de hardware y software \cite{patrones}. Este trabajo fue publicado en el Congreso Argentino de Sistemas Embebidos (CASE) en el año 2019 \cite{salt_case} y en IEEE en  2020 \cite{salt_paper}. \\\



Sin embargo, antes de pasar a una versión de producción, se determinó que faltaban implementar algunas funcionalidades adicionales para que se pueda instalar un SAL/T en las formaciones. Esta tesis aborda la implementación de esas funcionalidades, de acuerdo con lo establecido en el pliego de especificaciones técnicas para la versión completa del SAL/T escrito por la gerencia de seguridad operacional de Trenes Argentinos se encuentra disponible en \cite{spec}. \\
