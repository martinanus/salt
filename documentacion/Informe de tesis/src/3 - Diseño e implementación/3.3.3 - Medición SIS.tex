\subsubsection{Medición del estado de los SIS}


La medición del estado de los distintas señales de los SIS se realiza utilizando el conversor analógico digital (ADC) que trae el MCU; este permite la conversión de señales analógicas valores digitales que pueden ser procesados por el microcontrolador. En este caso, se utiliza el ADC3 del STM32, configurado para realizar conversiones en 10 canales diferentes. \\

La configuración del ADC se realizó para que opere con una resolución de 12 bits, lo que proporciona un rango de valores digitales de 0 a 4095, permitiendo una representación precisa de las señales analógicas; para lograr esta resolución, se consumen 15 ciclos del reloj de ADC. El prescaler del reloj se ajustó con un divisor de 8 lo que determina una operación a 10.5 MHz, y el modo de conversión en escaneo está habilitado para permitir la lectura de múltiples canales de manera secuencial. \\ 

El ADC está configurado para iniciar las conversiones mediante software, sin utilizar un disparador externo, lo que otorga mayor control sobre el inicio de las conversiones. Las conversiones no son continuas, y el DMA (Acceso Directo a Memoria) está deshabilitado de manera continua, permitiendo un mayor control del flujo de datos en el sistema. Además, los resultados se alinean a la derecha, lo que permite un procesamiento eficiente de los datos en 12 bits. \\

Se declara una función de callback HAL\_ADC\_ConvCpltCallback que se ejecuta al finalizar la 10ma conversión del ADC. En esta, se prende un flag de que hay una serie de valores medidos y se reinicia el proceso de adquisición de datos mediante DMA en los 10 canales configurados. El DMA permite que los datos del ADC se transfieran directamente a la memoria sin la intervención del procesador, lo que reduce la carga de la CPU y mejora el rendimiento. En modo no continuo, el ADC realiza conversiones de datos de manera intermitente o bajo demanda, activando el DMA solo cuando es necesario transferir una nueva muestra. \\

Cuando se corre la rutina de lectura de entradas del sistema, se verifica si hay una nueva serie de valores medidos. En ese caso, se compara para cada uno de los canales medidos (5 señales de CT y 5 de FE, una para cada SIS) con un valor umbral configurado dependiendo la tensión con la que trabaje el tren. Conociendo el circuito de hardware que reduce la tensión al rango de los 3V3 y sabiendo cuál es la tensión esperada en una llave abierta, se puede determinar qué valor es esperable encontrar en el ADC para una llave abierta. Para una llave cerrada, la tensión en sus terminales es cero y el valor que arroja el ADC también es cercano a 0. Por lo tanto, utilizando un umbral cercano al valor medio entre 0 y el esperado cuando la llave esté abierta, se puede determinar con el mejor margen el estado real de las señales del SIS. 