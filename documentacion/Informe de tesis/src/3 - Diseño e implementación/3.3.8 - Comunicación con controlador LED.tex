\subsubsection{Comunicación con controlador LED}

Para la configuración y utilización del controlador LED AS1115-BSST se utilizaron los detalles de los registros y formatos de escritura de la hoja de datos del producto. La comunicación con el SAL/T se realiza por el protocolo I2C. Al iniciar el controlador, se configuran los registros de modo de operación para configurar el modo normal de operación, se configura el registro que selecciona cuantos de los 8 dígitos disponibles a controlar se van a utilizar con el valor máximo, ya que se utilizan los primeros 4 dígitos para el display de velocidad y los otros, aunque no de manera completa, para alimentar los demás LEDs del panel frontal. También se configura la intensidad en su valor medio para obtener el brillo esperado. \\

El controlador trae opciones de decodificación para dígitos 7 segmentos en formato Code-B o HEX, pero no existe la posibilidad de configurarlo solo para los primeros 4 dígitos y utilizar los otros segmentos de manera individual. Por esto, es que se desactivó la decodificación en todos los dígitos y se dejó del lado del MCU la transformación de los números a los segmentos de cada dígito correspondientes.  \\ 


El SAL/T, en cada ciclo de operación, releva los valores correspondientes para el display de velocidad y para cada uno de los LEDs indicativos del panel frontal. Luego, transforma estos estados en los bits que corresponden en cada uno de los 8 registros utilizados del controlador LED y los envía por I2C para reflejar esos estados en las señales de activación de los LEDs en el panel frontal. 

