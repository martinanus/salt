\subsubsection{Mecanismos de RTOS}

El firmware del SAL/T está implementado utilizando FreeRTOS \cite{freertos} para permitir una ejecución pseudo-concurrente de las distintas tareas de manera eficiente, mejorando la respuesta del sistema ante eventos y mejorando el manejo de sus recursos. Se obtiene mayor control sobre la prioridad de las tareas y tiempos de respuesta, lo que lo hace ideal para aplicaciones de tiempo real críticas. Además, facilita la integración de sistemas complejos a través de las funcionalidades de asignación de tareas, comunicación entre tareas y manejo de memoria, siendo a la vez ligero y altamente personalizable para entornos de hardware específicos. \\ 


Se utilizó la interfaz de FreeRTOS-CMSIS versión 1.02, configurada con la unidad de punto flotante habilitada, permitiendo que las tareas de mayor prioridad interrumpan a las de menor prioridad (\textit{preemption}), asignando 15360 bytes de memoria para el tamaño total del \textit{heap}. \\

Las distintas \textit{tasks} o tareas utilizadas en el sistema son las siguientes: 

\begin{itemize}
    \item \textbf{commsTask - prioridad alta}: esta tarea se encarga de procesar los mensajes recibidos por cualquiera de las comunicaciones con dispositivos externos. Su prioridad es alta porque estas comunicaciones pueden traer datos que modifiquen completamente el estado del sistema y deben ser considerados lo antes posible para corregir el estado del sistema. 
    \item \textbf{systemTask - prioridad media}: esta tarea se encarga de leer de manera sistemática las entradas disponibles del sistema y luego de visualizarlas y registrarlas. 
    \item \textbf{modeTransTask - prioridad media}: esta tarea va a utilizar los últimos datos de entrada recibidos para determinar el modo de operación en el que se debe ejecutar el sistema.
    \item \textbf{criticalSigTask - prioridad media}: esta tarea se encarga de, basado en el modo de operación activo y en las demás entradas del sistema, determinar la activación o liberación de las señales críticas del sistema. 
    \item \textbf{reportTask - prioridad baja}: esta tarea reporta de manera periódica el estado del sistema de manera remota. Su prioridad es la más baja porque no modifica el comportamiento real del sistema y resulta no crítico el desfasaje en la transmisión. 
\end{itemize}


Las tareas de prioridad media ejecutan su rutina, y al terminar se pone un delay del sistema operativo de 500 ms y se le devuelve el control al \textit{scheduler} para que asigne los recursos de procesamiento a la tarea que siga en la cola de prioridades establecida. A ese delay se le suma una baja cantidad aleatoria de milisegundos para evitar la posible sincronización de las tareas. Por otro lado, la tarea de prioridad baja se ejecuta de manera periódica cada un tiempo configurable, por defecto 5 segundos. Las rutinas de todas las tareas se finalizan de manera rápida en pocos milisegundos, permitiendo que el asignador de tareas le dé el procesamiento a todas las tareas mientras se cumple el delay configurado para cada uno; de esta manera se evita lo que se conoce como \textit{starving} (que las tareas de menor prioridad nunca se ejecuten). \\ 

En cuanto a la tarea de mayor prioridad, se utilizó otro mecanismo de sincronización de FreeRTOS para facilitar el procesamiento de las comunicaciones; se utilizó un semáforo binario. Este mecanismo de sincronización tiene dos estados, disponible (1) o no disponible (0). Se utiliza para gestionar el acceso a un recurso compartido o para sincronizar tareas e interrupciones. Una tarea da o libera el semáforo cuando un evento ocurre, y otra tarea lo toma cuando está disponible; permitiendo la ejecución controlada de procesos dependientes. \\

En el SAL/T, todas las comunicaciones con sistemas externos (módulo GPS, módulo Wi-Fi, fuentes de medición de velocidad, conexión serie local) están conectadas por interfaz UART al MCU y tienen habilitada la recepción por interrupción. Esta interrupción por hardware tiene mayor prioridad que las tareas del kernel lo que permite interrumpir cualquier tarea que se esté ejecutando. En la función de \textit{callback}, se mantienen buffers de recepción separados (uno por interfaz) hasta detectar la recepción de una línea completa de un mensaje. En ese momento, se almacena la línea completa en una variable global, accesible desde la rutina de las comunicaciones, y se libera el semáforo de comunicaciones. \\

Al terminar la interrupción por hardware, el asignador de tareas reprioriza las tareas y en caso de haber liberado el semáforo, le pasa el control a la tarea de mayor prioridad. En esta tarea se verifica qué comunicación tiene un mensaje nuevo y allí se procesan los datos correspondientes. Al finalizar, se vuelve a tomar el semáforo bloqueando la tarea hasta la recepción de una nueva línea de información. 

 