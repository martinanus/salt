\subsubsection{Procesamiento de velocidades por RS485}

Para la comunicación con las fuentes externas de medición de velocidad, se utilizaron las interfaces UART7 y UART8 del MCU que, a través de un circuito conversor de UART a RS485, logra comunicarse con los sistemas externos. La configuración de ambos UART establece la comunicación serie con una velocidad de 115200 baudios, utilizando 8 bits de datos, 1 bit de parada y sin paridad. El modo está configurado sin control de flujo por hardware, y con sobremuestreo a 16, lo que asegura una comunicación eficiente y precisa. \\

La interfaz RS485 1, conectada con la señal interceptada entre el Hasler Teloc 1500 y el indicador de velocidad de la formación, recibe el dato de la velocidad encapsulado dentro de un paquete de 31 bytes. Este paquete comienza y termina siempre por el byte 0x7E lo que permite delimitar la trama. Esta interfaz trabaja con la recepción por interrupción y al detectar el byte de inicio comienza almacenar la información recibida y al recibir el byte de stop, validando una longitud de 31 bytes totales, termina la línea de datos. En ese momento, se libera el semáforo para que el RTOS permita la ejecución de la rutina de procesamiento de esta línea donde se extrae de los bytes 7 y 8 la información de la velocidad. Nuevamente se incluye una marca de tiempo para verificar la validez del dato antes de ser utilizado en otras rutinas. \\

La interfaz RS485 2, conectada con la medición de velocidad proveniente del generador de impulsos ópticos, recibe el dato de la velocidad de manera directa. Se espera la recepción de un mensaje numérico seguido de un carácter de retorno de línea. Esta interfaz, conectada al UART8, funciona utilizando la recepción por interrupción y mediante un \textit{buffer} que almacena cada carácter hasta encontrar la finalización de la línea. Una vez completa la línea, se activa el semáforo para que al finalizar la rutina de interrupción, el RTOS pueda darle mayor prioridad al procesamiento de esa línea donde se va a intentar convertir los caracteres leídos a un valor numérico y almacenarlo como dato de velocidad. También se almacena una marca de tiempo para asociarla al dato de velocidad y permitir la expiración de este dato tras algunos segundos sin recepción de nuevos valores. 