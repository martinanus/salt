\subsubsection{Interacción con datos GPS}

El módulo GPS utilizado en el sistema se comunica mediante interfaz UART con el MCU del sistema. Se configuró la UART5 para esta comunicación, con un \textit{baudrate} de 9600, una longitud de palabra de 8 bits, sin bit de paridad y un bit de stop conocido como configuración 8N1. No se emplea control de flujo por hardware y el sobremuestreo está configurado en 16, lo que mejora la precisión en la recepción de datos. \\ 

El módulo GPS transmite datos a una frecuencia de 1 Hz, es decir, una vez por segundo, enviando múltiples líneas de información conocidas como frases NMEA (\textit{National Marine Electronics Association}). Estas sentencias incluyen detalles sobre la posición (latitud y longitud), velocidad, altitud, fecha, hora y estado de los satélites \cite{nmea}. En este proyecto, se utiliza la línea de GPRMC (\textit{Recommended minimum specific GPS/Transit data}); una de las más importantes en la transmisión de datos GPS. Contiene información esencial como la hora, fecha, posición (latitud y longitud), velocidad sobre el suelo y dirección, y es utilizada para obtener los datos mínimos necesarios para la navegación. \\ 

La recepción de la información se realiza utilizando la comunicación por interrupción donde un \textit{buffer} receptor permite identificar el comienzo de esta frase porque siempre comienza por los caracteres \$GPRMC y termina por el carácter de retorno de línea. Una vez leída una línea completa, se activa el semáforo de procesamiento de comunicaciones y una vez terminada la interrupción de hardware de la UART, el RTOS va a priorizar el procesamiento de esta línea. Se almacena una marca de tiempo de la recepción de este dato para más adelante poder verificar la validez o expiración de la información si no se recibieron nuevos datos. La línea trae definida el orden de los datos separados por una coma. De esta manera, se generó un \textit{parser} que permite asignar a una estructura los distintos datos presentes en la frase GPRMC. \\ 

El dato más relevante a consultar es el estado de la transmisión. Si este dato indica que la conexión está ausente (probablemente por falta de señal de la antena con los satélites), todos los otros datos deben ser descartados. En caso de leer un estado de conexión exitoso, se determinan los datos de latitud, longitud y velocidad transmitidos por el módulo. \\
