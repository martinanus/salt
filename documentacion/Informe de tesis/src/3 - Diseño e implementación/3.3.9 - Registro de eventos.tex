\subsubsection{Registro de eventos}

El SAL/T tiene la capacidad de registrar eventos tanto de manera local como de manera remota. Localmente, interactúa con la tarjeta SD conectada por interfaz SPI y ante cada evento registrable, se escribe dentro de un archivo de logueo donde se va acumulando el histórico de eventos. A nivel remoto, para cada evento, se utiliza la comunicación UART con el módulo ESP32 para que este luego retransmita el mensaje al broker MQTT en el \textit{topic} de logs, y la central operativa pueda recibir estos eventos en tiempo real y almacenarlos en una base de datos propia. \\

Todos los eventos registrados están acompañados de una marca de tiempo que lleva internamente el SAL/T. Cada 10 segundos, el ESP32 envía una actualización de la fecha y hora para que el SAL/T se pueda sincronizar en caso de reinicio o eventual desfasaje del reloj interno. \\ 

El SAL/T registra cualquier evento de cambio de estado que pueda detectar. Algunos de ellos son los siguientes: 
\begin{itemize}
    \item Detección de alimentación tras reinicio
    \item Cambio de estado en la medición de algún SIS
    \item Cambios de estado en la activación del aislamiento de algún SIS
    \item Cambio de estado de modo del sistema
    \item Cambio de zona de circulación
    \item Cambio de estado de la conexión GPS
    \item Cambio de fuente de medición de velocidad
    \item Cambio de velocidad en modo aislado limitado    
    \item Aplicación de cambios de configuración tras recepción de comando remoto
    \item Aplicación de comando remoto
\end{itemize}

El SAL/T permite la descarga de todos los datos registrados en la tarjeta SD de manera local, conectándose por USB a una comunicación serie 8N1, \textit{baudrate} 115.000 y enviando el comando DOWNLOAD\_LOGS. Como respuesta, se envían secuencialmente todos los datos almacenados en la memoria. Esta opción también es posible de manera remota, enviando desde la central operativa el mismo comando y recibiendo la respuesta por el canal de logs del broker MQTT. 