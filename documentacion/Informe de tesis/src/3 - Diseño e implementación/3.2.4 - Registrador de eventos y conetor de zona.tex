\subsubsection{Registrador de eventos y conector de zona}

Las formaciones ferroviarias cuentan con un registrador de eventos que es un dispositivo que monitorea y almacena datos clave sobre el funcionamiento del tren, similar a una caja negra en aviones \cite{registrador_eventos}. Registra información como la velocidad, la activación de frenos, la posición del acelerador, el estado de las señales y otros parámetros críticos durante el viaje. Estos datos son fundamentales para analizar el rendimiento, investigar accidentes o fallos, y mejorar la seguridad operativa en el sistema ferroviario. En las formaciones para la cual se diseñó el SAL/T, se consideró la interacción con un registrador de eventos modelo Hasler Teloc 1500 \cite{hasler}. \\ 

Para el sistema SAL/T se solicitó la comunicación del estado de las siguientes características: 

\begin{itemize}    
    \item Estado de la alimentación del SAL/T
    \item Activación del modo limitado
    \item Activación del freno de emergencia
    \item Activación del corte de tracción
\end{itemize}

La comunicación con el registrador de eventos se debe realizar a través un camino de baja impedancia entre dos nodos. Para esto, se utilizó nuevamente los relés JW2SN-DC5V con un nuevo ULN2003 como driver de potencia para estos 4 relés. \\

Otra comunicación de estado que realiza el SAL/T mediante un 5to relé, también conectado al mismo driver de potencia, es con un conector de zona. Este relé debe activar un camino de baja o alta impedancia dependiendo de la zona en la que circule el tren. La ubicación es obtenida mediante GPS y, cuando se supera una distancia configurada respecto al punto de la estación de origen del recorrido, se activa un camino de baja impedancia. Esto activa en la formación un cambio en los límites de velocidad de circulación y es utilizado para formaciones que tienen parte de su recorrido en zonas rurales o menos urbanizadas. De esta manera, permite mantener distintos límites de velocidad dependiendo la zona de circulación de la formación. 