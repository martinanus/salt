\section{Anexo}\label{anexo}

\subsection{Detalle de requerimientos}

\begin{enumerate}
    \item Grupo de requerimientos asociados con la interfaz humano-máquina
    \begin{enumerate}
        \item La interfaz debe contar con una llave rotativa para activar el modo aislado limitado.
        \item La interfaz debe mostrar la velocidad media del equipo en km/h con 4 dígitos.
        \item La interfaz debe indicar el estado de la señal de corte de tracción.
        \item La interfaz debe indicar el estado de la señal de freno de emergencia.
        \item La interfaz debe indicar la presencia de un comando remoto de la central operativa.
        \item La interfaz debe indicar el estado del módulo GPS.
        \item La interfaz debe indicar el estado de la alimentación.
        \item La interfaz debe indicar el modo en el que se encuentra el sistema.
        \item La interfaz debe contener un control local por fuera del panel frontal para activar el modo aislado total.
        \item La interfaz debe indicar con un LED RGB en cuál de las 3 zonas geográficas predefinidas se encuentra el material rodante. 
        \item La interfaz debe indicar el estado de los distintas Sistemas Instrumentados de Seguridad conectados al SAL/T con un LED RGB para cada SIS.
        \item La interfaz debe contar con una botonera o switch que permita cambiar el perfil para modo de tracción intermitente.
        \item La interfaz debe indicar el perfil seleccionado para modo de tracción intermitente.
        \item La interfaz debe contener un buzzer interno para indicaciones sonoras.
    \end{enumerate}
       
    \item Grupo de requerimientos asociados a la comunicación con el registrador de eventos
    \begin{enumerate}
        \item El sistema debe informar al registrador de eventos la activación del modo aislado limitado.
        \item El sistema debe informar al registrador de eventos si la alimentación es correcta.
        \item El sistema debe informar al registrador de eventos la activación del freno de emergencia.
        \item El sistema debe informar al registrador de eventos la activación del corte de tracción.
    \end{enumerate}
       
    \item Grupo de requerimientos asociados al registro de datos
    \begin{enumerate}
        \item El sistema debe mantener un registro de datos local donde se registran los siguientes eventos con marca de tiempo:
        \begin{itemize}
            \item Activación/Desactivación del modo aislado limitado
            \item Activación/Desactivación del modo aislado total (local o remota)
            \item Referencia de velocidad adoptada y sus permutaciones 
            \item Cobertura de señal GPS
            \item Ciclos de permiso, corte y freno en modo de tracción intermitente
            \item SIS en fallo
            \item Velocidad desarrollada
            \item Zona de circulación
        \end{itemize}
        \item El sistema debe permitir descargar los datos registrados mediante una interfaz serie 
        \item El sistema debe poder descargar los datos registrados de manera remota 
    \end{enumerate}
       
    \item Grupo de requerimientos asociados a la comunicación con la central operativa
    \begin{enumerate}
        \item El sistema debe informar periódicamente (con un tiempo configurable) su estado a la central operativa a través de la red de datos WiFi.
        \item Debe existir la posibilidad de usar 2 proveedores distintos de datos de manera simultánea.
        \item Debe existir la posibilidad de conectarse alternadamente a más de una red Wifi
        \item El protocolo de comunicación con la central operativa debe ser MQTT.
        \item El sistema debe ser capaz de recibir un comando remoto que anule el corte de tracción y el freno de emergencia bajo cualquier condición (modo aislado total).
        \item El sistema debe ser capaz de recibir un comando remoto que active el corte de tracción y el freno de emergencia bajo cualquier condición (modo parada total).
        \item El sistema debe ser capaz de recibir un comando remoto que active el corte de tracción y anule el freno de emergencia bajo cualquier condición (modo coche en deriva).
        \item El sistema debe ser capaz de recibir un comando remoto que active el corte de tracción y el freno de emergencia de forma intermitente en ciclos de tiempo configurables (modo intermitente).
        \item El sistema debe ser capaz de recibir un comando remoto que cancele cualquier comando remoto vigente.
        \item El sistema debe ser capaz de recibir comandos remotos que modifiquen sus parámetros internos configurables.
        \item Si no se recibe un nuevo comando remoto luego de un tiempo configurable (por defecto 10 segundos, máximo 1 minuto), debe volver al algoritmo de activación de corte de tracción y freno de emergencia por defecto.
        \item Ante un comando remoto recibido, debe enviar una confirmación de recepción que permita a la central operativa decidir si es necesaria o no una retransmisión.
        \item Debe utilizar algún mecanismo de encriptación para el enlace con la central operativa.
    \end{enumerate}
       
    \item Grupo de requerimientos asociados al modo normal de funcionamiento
    \begin{enumerate}
        \item El modo normal el sistema no debe intervenir en el funcionamiento del material rodante (prioridad alta).
        \item El sistema debe obtener en todo momento la mejor estimación posible de la velocidad de la formación.
        \item Debe ser capaz de recibir la velocidad a partir de una señal digital provista por el registrador de eventos Hasler Teloc 1500
        \item Debe ser capaz de calcular la velocidad a partir de un sistema GPS integrado.
        \item Debe ser capaz de calcular la velocidad a partir de un generador de impulsos
        \item El rango de velocidad soportado por el sistema tiene que estar entre 0 y 120 km/h.
        \item La estimación de velocidad debe tener una precisión del 2 % de fondo de escala.
        \item El sistema debe determinar la posición del material rodante para activar un contacto seco de relé según la zona de operación donde se encuentre.
    \end{enumerate}
       
    \item Grupo de requerimientos asociados al modo aislado limitado
    \begin{enumerate}
        \item En modo aislado limitado el sistema debe evitar la aplicación del corte de tracción.
        \item En modo aislado limitado el sistema debe evitar la aplicación del freno de emergencia.
        \item Ante cualquier error interno, el sistema debe dejar de intervenir en la aplicación del corte de tracción.
        \item Ante cualquier error interno, el sistema debe dejar de intervenir en la aplicación del freno de emergencia.
        \item En modo aislado limitado el sistema debe emitir una señal sonora intermitente a través de un buzzer.
        \item En modo aislado limitado el sistema debe evitar que la velocidad del material rodante supere una serie de límites configurados.
        \begin{enumerate}
            \item Si al pasar de modo normal a modo aislado limitado no se cuenta con una estimación de velocidad, debe activar el corte de tracción y el freno de emergencia por 30 segundos.
            \item Si se supera una velocidad configurable (por defecto 30 km/h), debe activar el corte de tracción y emitir una señal sonora continua a través de un buzzer.
            \item Si se supera una velocidad configurable (por defecto 36 km/h), debe activar el freno de emergencia.
            \item na vez aplicado, el corte de tracción debe dejar de aplicarse si la velocidad vuelve a ser menor a una velocidad configurable (por defecto 25 km/h).
            \item Una vez aplicado, el freno de emergencia sólo debe dejar de aplicarse luego de un tiempo configurable (por defecto 30 segundos) desde que se superó el límite.
            \item Si la lectura de velocidad es inválida, debe activar y desactivar el corte de tracción y freno de emergencia de manera alternada en ciclos de tiempo según perfiles preconfigurados (modo de tracción intermitente).
        \end{enumerate}
        \item El sistema debe interactuar con 5 Sistemas Instrumentados de Seguridad.
        \item El sistema debe ser capaz de identificar qué SIS está fallando y aislar solamente este dejando los demás SIS operativos.
        \item El sistema debe medir la posición del material rodante para modificar las velocidades máximas según la zona de operación donde se encuentre.
    \end{enumerate}
       
    \item Grupo de requerimientos asociados al modo aislado total
    \begin{enumerate}
        \item El sistema solo puede acceder a este modo accionando el control local o mediante comando remoto.
        \item En el modo aislado total el sistema debe liberar la velocidad de circulación.
        \item El sistema debe registrar en el registro de datos la activación de este modo.
    \end{enumerate}
       
    \item Grupo de requerimientos asociados al modo de tracción intermitente
    \begin{enumerate}
        \item El sistema solo deberá acceder a este modo cuando se encuentre en modo aislado limitado y no cuente con ninguna medición de velocidad válida.
        \item El sistema debe permitir la configuración remota de distintos perfiles de operación de manera remota en donde se configuran los valores de Tiempo del permiso de Tracción, Tiempo de ciclo en deriva, Cantidad de ciclos antes de aplicar freno y Tiempo de espera antes de comienzo de nuevo ciclo.
    \end{enumerate}
       
    \item Grupo de requerimientos asociados al hardware y al gabinete
    \begin{enumerate}
        \item El sistema utilizará una fuente externa de 5 V de tensión continua que pueda soportar hasta 25W de potencia.
        \item Los conectores deben estar bien identificados para facilitar la correcta conexión
        \item El sistema debe poseer una placa de circuito impreso con el procesador y periféricos necesarios para el procesamiento de las señales del material rodante y una segunda placa para el los elementos necesarios para la interacción con el conductor
        \item El diseño de la placa de circuito impreso debe contemplar las recomendaciones de la norma IPC 2221 a fines de asegurar la manufacturabilidad del equipo. 
    \end{enumerate}
           
    \item Grupo de requerimientos asociados al desarrollo del firmware
    \begin{enumerate}
        \item El desarrollo del software debe contemplar las recomendaciones de las normas UNE-EN 50126 y UNE-EN 50128.
        \item El sistema debe trabajar con un sistema operativo en tiempo real para mejorar el desempeño y la seguridad del equipo.
    \end{enumerate}
\end{enumerate}