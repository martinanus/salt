
\section{Referencias}


 \begin{thebibliography}{30}

\bibitem{norma_50126}
\textbf{UNE-EN 50126-1}. Aplicaciones ferroviarias. Especificación y demostración de la fiabilidad, la disponibilidad, la mantenibilidad y la seguridad (RAMS). 2005. \href{https://www.une.org/encuentra-tu-norma/busca-tu-norma/norma?c=N0033106}{https://www.une.org/encuentra-tu-norma/busca-tu-norma/norma?c=N0033106}

 \bibitem{gicasfe}
 \textbf{GICSAFe} Grupo de Investigación en Calidad y Seguridad de las Aplicaciones Ferroviarias. \href{https://sites.google.com/view/conicet-gicsafe}{https://sites.google.com/view/conicet-gicsafe}.


 \bibitem{trenes_arg}
 \textbf{Trenes Argentinos} Trenes Argentinos Operaciones, empresa del estado. \href{https://www.argentina.gob.ar/transporte/trenes-argentinos}{https://www.argentina.gob.ar/transporte/trenes-argentinos}.

 \bibitem{salt_ivan}
 \textbf{Di Vito, Iván Mariano, “Aplicación de la técnica de patrones de diseño a la implementación de un Sistema de Aislamiento Limitado/Total ferroviario,”} 2019 \href{https://bibliotecadigital.fi.uba.ar/items/show/18446}{\textit{https://bibliotecadigital.fi.uba.ar/items/show/18446}}.

 \bibitem{patrones}
 \textbf{Ashraf Armoush. “Design Patterns for Safety-Critical Embedded Systems”} RWTH Aachen University. 2010.

 \bibitem{salt_case}
 \textbf{Ivan Mariano Di Vito, Pablo Gomez, Ariel Lutenberg, Adrian Laiuppa “Sistema de supervisión de la seguridad del material ferroviario utilizando patrones de diseño”.} 2019. \href{https://drive.google.com/file/d/1FDtRL4We4daZ7XrJ81sL2f6HrZbYlOuG/view}{\textit{https://drive.google.com/file/d/1FDtRL4We4daZ7XrJ81sL2f6HrZbYlOuG/view}}. 

 
 \bibitem{central_op}
 \textbf{Iglesias Fernando Julio, Sambrizzi Matias,  “Central Operativa para el Sistema de Aislamiento Limitado/Total Ferroviario (SAL/T)“}. Plan de trabajo, 2022. 

 \bibitem{plan_fiuba}
 \textbf{Plan de estudios de Ingeniería Electrónica - Facultad de Ingeniería, Universidad de Buenos Aires}. Plan 2009, modificación 2018. \href{https://cms.fi.uba.ar/uploads/ELECTRONICA_Modificacion_2018_Plan_2009_Ingenieria_Electronica_RCS_1801_18_f22d3d7a5a.pdf}{\textit{https://cms.fi.uba.ar/uploads/ELECTRONICA\_Modificacion\_2018\_Plan\_2009\_Ingenieria\_ Electronica\_RCS\_1801\_18\_f22d3d7a5a.pdf}}. 


 \bibitem{spec}
 \textbf{SISTEMA DE AISLADO LIMITADO / TOTAL (SAL-T)}. ESPECIFICACIÓN TÉCNICA. Bypass Sistemas Instrumentados de Seguridad a Bordo del Material Rodante ET.SO. No 046 /18 – E3. 2022.


 \end{thebibliography}