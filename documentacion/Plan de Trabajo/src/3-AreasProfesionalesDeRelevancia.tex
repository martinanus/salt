\section{Áreas profesionales de relevancia}

Esta Tesis está enmarcada principalmente dentro del área de Sistemas Digitales y Computación, según lo especificado en la versión 2009, Modificación 2018, del plan de estudios de la Carrera de Ingeniería Electrónica de la Facultad de Ingeniería de la Universidad de Buenos Aires \cite{plan_fiuba}. \\

En particular, están vinculadas a este proyecto las siguientes actividades reservadas al tıitulo de Ingeniero Electrónico:
\begin{itemize}
    \item Diseñar, proyectar y calcular hardware y software de sistemas embebidos y dispositivos lógicos programables
    \item Proyectar, dirigir y controlar la construcción, implementación, mantenimiento y operación de lo mencionado anteriormente.
\end{itemize}


De la misma forma, son particularmente relevantes al proyecto los siguientes alcances del titulo de Ingeniero Electrónico: 

\begin{itemize}
    \item Estudio, planificación, proyectos, estudios de factibilidad técnico-económicos, programación, dirección, construcción, instalación, puesta en marcha, operación, ensayo, mediciones, mantenimiento, reparación, modificación, transformación e inspección de:
    \begin{itemize}
        \item Sistemas, subsistemas, equipos, componentes, partes, piezas (Hardware), de procesamiento electrónico de datos en todas sus aplicaciones incluyendo su programación (Software) asociada.
        \item Sistemas, subsistemas, equipos, componentes, partes, piezas que impliquen electrónica, de navegación o señalización o cualquier otra aplicación al movimiento de vehículos terrestres, aéreos, marítimos o de cualquier otro tipo.

    \end{itemize}
\end{itemize}
