\section{Alcance}



El proyecto consiste en diseñar, desarrollar e implementar un nuevo protitpo del SAL/T teniendo en cuenta que se deben agregar nuevas funcionalidades requeridas, que el hardware debe ser actualizado con componentes vigentes y disponibles actualmente en el mercado internacional y que el firmware desarrollado debe estar basado en un sistema operativo de tiempo real que resulte portable. \\

Para establecer los requerimientos del proyecto, se contempló el prototipo preexistente del SAL/T y el pliego de especificaciones técnicas que generó Trenes Argentinos teniendo en consideración las limitaciones de tiempos y presupuesto que hace sentido para un proyecto de fin de carrera de grado. En el \nameref{anexo}, se adjunta un listado con el detalle de los requerimientos considerados para este proyecto.\\

Una vez finalizado el diseño del sistema, el prototipo debe ser fabricado sobre una placa PCB y se deben montar todos los componentes para su correcto funcionamiento. Se deberán realizar pruebas de laboratorio en las condiciones de funcionamiento para verificar que se cumplas todos los requerimientos establecidos. De ser posible, se intentará coordinar con Trenes Argentinos para realizar algunas pruebas de campo en los talleres de la empresa. 

