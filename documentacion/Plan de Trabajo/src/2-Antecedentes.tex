\section{Antecedentes}

El Grupo de Investigación en Calidad y Seguridad de las Aplicaciones Ferroviarias (GICSAFe) fue creado en 2017 en el marco del Consejo Nacional de Investigaciones Científicas y Técnicas (CONICET) de la República Argentina \cite{gicasfe}. En el grupo participan investigadores, docentes y alumnos de diferentes instituciones públicas argentinas quienes realizan desarrollos de sistemas electrónicos e informáticas para aplicaciones ferroviarias relacionadas con la seguridad. Muchos de los prototipos desarrollados son instalados y entregados junto a toda la documentación, respetando normas internacionales de seguridad, para luego transferir el derecho de uso, fabricación y mantenimiento a los clientes. \\

La empresa estatal Trenes Argentinos Operaciones \cite{trenes_arg} encargó al grupo CONICET-GICSAFe el desarrollo de un prototipo del SAL/T en el año 2017. El ingeniero Iván Di Vito, en ese momento estudiante de la carrera de Ingeniería Electrónica en la UBA y allegado al grupo de investigación tomó el proyecto que finalizaría en el año 2019 con la entrega de un prototipo funcional del SAL/T luego de varias pruebas de campo en los talleres de Trenes Argentinos junto a toda la documentación requeridas por las normas de seguridad ferroviarias para este tipo de desarrollos \cite{salt_ivan}. El trabajo realizado incluyó el relevamiento de los requerimientos del proyecto, el desarrollo conceptual de la solución propuesta, el diseño e implementación del hardware y firmware, la puesta en marcha y realización de pruebas de campo de un prototipo funcional. El mismo se realizó de acuerdo a la norma UNE-EN 50126, aplicando técnicas de
patrones de diseño de hardware y software \cite{patrones}. Este trabajo fue publicado por el Congreso Argentino de Sistemas Embebidos (CASE) en el año 2019 en la categoría de artículo \cite{salt_case}. \\

En el año 2021, otros dos estudiantes de la carrera de Ingeniería Electrónica de la UBA, Fernando Iglesias y Matias Sambrizzi, emprendieron un proyecto para el desarrollo de la central operativa del SAL/T en el mismo contexto  de acuerdo entre Trenes y CONICET-GICSAFe. El objetivo de dicho proyecto fue desarrollar el software de una central operativa que permita administrar y configurar
de forma remota dispositivos de supervisión de seguridad de formaciones ferroviarias SAL/T \cite{central_op}. Para esto, se implementó un servicio de monitoreo y control que permite visualizar en tiempo real los datos recibidas y enviar comando de control a los dispositivos activos, la gestión de configuración de dispositivos, la base de datos, los componentes de seguridad necesarios para las comunicaciones e información almacenada y la gestión de usuarios y perfiles. \\
